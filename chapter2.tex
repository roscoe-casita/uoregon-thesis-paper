\chapter{Definitions}

\section{Lists, Queues, Stacks, N-Trees, etc...}
This paper assumes the reader is familiar with the basic data structures such as lists, arrays, stacks, queues, trees, graphs etc. Let a list $l$ be an ordered list of things $\{t,i\}$ where each thing $t$ must be the same type, and $t$ is only distinguishable by its index $i$. For the purposes of this paper, $x.push(y)$ will insert $y$ at the last index of the list, $y \gets x.pop()$ will remove from the last index of the list and assign it to the variable $y$.  $x.enque(y)$ will insert $y$ at the last index of the list. $y \gets x.dequeue()$ will remove from the first index of the list and assign it to the variable $y$. In these example $x$ has an associated type, $y$ is a value of that same type in all cases. A list-of-lists structure is used in this paper to traverse an N-way branching tree. The internal C++ implementation is an array representation with $O(1)$ index lookup time, $O(N)$ seek, insert, and remove times.

\section{Odometers }

An odometer is list of integer numbers. An odometer is a construct used extensively throughout this paper as the indexes of items in a list is fundamental to proving odometer equivalence to a hyperedge. As an unrestricted list of numbers the odometer is similar to an instance of a Turing machine \textit{tape}. \\

Let an odometer $o$ be an list of integers $n$ and indexes $\{n,i\}$. The $i^{th}$ indexable integer of an odometer can be written $n_i = o[i]$. Integers $n$ can be repeated, they are distinguished via their index. Indexes $i$ are unique non-repeating whole numbers from $[0,\infty]$. The size of the odometer is written as $o.size()$, is the count of $\{n,i\}$. \\

Every integer can take on all values from $[-\infty,\infty]$, where $\infty$ is a function of the number of bits on the system. There are $N$ indexable integers in every odometer. Thus for every different $N$ there are are $N^M$ distinctly different odometers instances that can be created.\cite{Odometer:Fuchs}\\


The following common functions are defined in the Code Appendix: $Union$, $Intersection$, $Minus$, $StrictEqual$, $SetEqual$. The additional trivial functions are implemented in terms of odometers and/or lists of odometers: $DoesACoverB$, $DoesAHitB$,$DoesACoverBorBCoverA$, $DoesAHitAll$, $DoesAnyHitA$, $GenerateNMinusOne(o)$. Please note that all functions are polynomial in both space and time.\\

\section{Hypergraphs}
\subsection{Unrestricted Hypergraphs}
The traditional hypergraph definition $H=(V,E)$ is terse for implementers. Traditionally a hypergraph is defined as a collection of sets where there is no ordering and repeated elements are not allowed. The following definitions were used  to implement the hypergraph interface. The odometer is of particular interest as it can be used independently from hypergraphs for linear integer optimization techniques.\\

Let a hyperedge $e$ be a list of vertexes: $e =\{v,i\}$. The $i^{th}$ indexable vertex of $e$ can be written $v_i = e[i]$. Vertexes $v$ can be repeated, they are distinguished via their index. Indexes $i$ are unique non-repeating whole numbers from $[0,\infty]$. The size of the hyperedge written as $e.size()$ is the count of $\{v,i\}$.


Let an unrestricted hypergraph $U$ be a single hyperedge $nodes$ and the two functions $OtoE$ and $EtoO$. $OtoE$ is the surjective function to map a given odometer to a hyperedge. $EtoO$ is the injective function to map a given hyperedge to an odometer. The hyperedge $U.nodes$ cannot repeat any vertexes $v$ for the function $EtoO$ to behave correctly. \\

Given these definitions, the following is now possible given a hypergraph: A hyperedge can be constructed from an odometer. An odometer can be constructed from a hypergraph. While the functions in the paper use hyperedges the code uses odometers in place of hyperedges. Thus every instance of a hyperedges can be converted to an instance of an odometer, and every instance of an odometer can be converted to an instance of a hyperedge.\\

Specifically the odometer is an instance of a set of integer numbers that can be reasoned about independently of a hypergraph. The code implements some common set functions that allow the constraints to be expressed, such as union, minus, include short circuit versions of functions for faster performance. 

\begin{algorithm}
	\caption{OdometerToHyperedge}\label{OtoE}
	\begin{algorithmic}[1]
		\Function{OtoE($U,o$)}{}
		\State $e \gets \emptyset$
		\State $size \gets U.nodes.size()$
		\ForAll {$ \{n,i\} \in o$}
		\State // where -2 mod 7 = 5 .
		\State $e[i] \gets U.nodes[ n \% size ]$ // convert an integer number to index.
		\EndFor
		\State \Return $e$
		\EndFunction
	\end{algorithmic}
\end{algorithm}
\begin{algorithm}
	\caption{HyperedgeToOdometer}\label{EtoO}
	\begin{algorithmic}[1]
		\Function{EtoO($U,e$)}{}
		\State $o \gets \emptyset$
		\ForAll{$\{v_e,i_e\} \in e$}  // use hashmap of v to i
		\ForAll{$\{v_n,i_n\} \in U.nodes$} // to reduce $O(n^2)$ to $O(n)$
		\If {$v_e = v_n$}
		\State $o[i_e] \gets i_n$ // lookup index and save as integer.
		\EndIf
		\EndFor
		\EndFor
		\State \Return $o$
		\EndFunction
	\end{algorithmic}
\end{algorithm}

Notice that these functions provide polynomial time access to all permutations, combinations, repeats, patterns etc. Thus reasoning about a hyperedge is equivalent to reasoning about its corresponding odometer, and vice versa. Vertex data can be complex and large, thus reasoning about the odometer in place of the hyperedge is for performance and interesting reasons noted later.
\newpage

\subsection{Normal Hypergraphs }
Let a \textit{normal} hypergraph \cite{Hypergraph:Book} be $H = (V,E)$ where $V$ is a list of vertexes $\{v,i\}$, $E$ is a list of hyperedges $\{e,i\}$ where each hyperedge $e$ is a sublist of $V$. The following trivial restrictions must be imposed to get the expected behavior out of a \textit{normal} hypergraph given the unrestricted list definitions. No hyperedge contains a duplicated vertex. Every vertex in all hyperedges is contained in the hypergraph list of vertexes. There are no duplicate hyperedges. The maximal size of a hyperedge is the size of all hypergraph vertexes. Every vertex exists in at least one hyperedge. There are no duplicate vertexes in the hypergraph.\\


\begin{equation*}
\begin{matrix}
\forall e \in E , \forall v, v' \in e \vert v \ne v'\\
\forall e \in E , \forall v \in e \vert v \in V\\
\forall e \in E , \not\exists e' \in E \vert e = e'\\
\forall e \in E \vert |e| \le |V|\\
\forall v \in V , \exists e \in E \vert v \in e\\
\forall v \in V , \not\exists v' \in V \vert v = v'\\
\end{matrix}
\end{equation*}

\subsection{Simple Hypergraphs}
Let a $simple$ hypergraph be $H=(V,E)$ as $normal$ hypergraph with the additional restriction that no hyperedge fully contains any other hyperedge. \begin{equation*}
\begin{matrix}
\forall e , e' \in E \vert e \not \subseteq e' \wedge e' \not \subseteq e\\
\end{matrix}
\end{equation*}

\section{Minimal Transversal of a Hypergraph}
Let the transversal of a hypergraph $T \subseteq H.V$ be a hitting set of all the hyperedges of a hypergraph such that $DoesAHitAll(T,H.E) = true$. Using the definitions of $GenerateNMinusOne$ the following implementation determines if an odometer hits ever odometer in a list.\\

\begin{algorithm}
	\caption{IsMinimalTransversal}\label{IsMinimalTransversal}
	\begin{algorithmic}[1]
		\Function{IsMinimalTransversal($o,list\_of\_o$)}{}
		\If {$DoesAHitAll(o,list\_of\_o)=false$}
		\State \Return $false$
		\EndIf
		\ForAll {$\{o_n,i_n\} \in GenerateNMinusOne(o)$}
		\If {$DoesAHitAll(o_n,list\_of\_o)$}
		\State \Return $false$
		\EndIf
		\EndFor
		\State \Return $true$
		\EndFunction
	\end{algorithmic}
\end{algorithm}


\section{All minimal transversals of a hypergraph}
There are $2^{|V|}$ possible combination sets that can be derived from the hypergraph $H=(V,E)$ and therefore $2^{|V|}$ transversals that need to be enumerated in the worst case. Tractable scalable algorithms fundamentally need to use polynomial space storage and exponential time to enumerate the traversals efficiently.
