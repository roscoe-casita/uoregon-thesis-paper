\chapter{Odometers}


\section{Introduction}

\section{Odometers }

An odometer is list of integer numbers. An odometer is a construct used extensively throughout this paper as the indexes of items in a list is fundamental to proving odometer equivalence to a hyperedge. As an unrestricted list of numbers the odometer is similar to an instance of a Turing machine \textit{tape}. \\

\begin{definition}
Let an odometer $o$ be an list of integers $n$ and indexes $\{n,i\}$. The $i^{th}$ indexable integer of an odometer can be written $n_i = o[i]$. Integers $n$ can be repeated, they are distinguished via their index. Indexes $i$ are unique non-repeating whole numbers from $[0,\infty]$. The size of the odometer is written as $o.size()$, is the count of $\{n,i\}$. \\
\end{definition}

Every integer can take on all values from $[-\infty,\infty]$. There are $N$ indexable integers in every odometer. Thus for every different $N$ there are are $N^{\infty}$ distinctly different odometers instances that can be created. \cite{Odometer:Fuchs}\\


The following common functions are defined in the Code Appendix: $Union$, $Intersection$, $Minus$, $StrictEqual$, $SetEqual$. The additional trivial functions are implemented in terms of odometers and/or lists of odometers: $DoesACoverB$, $DoesAHitB$,$DoesACoverBorBCoverA$, $DoesAHitAll$, $DoesAnyHitA$, $GenerateNMinusOne(o)$. Please note that all functions are polynomial in both space and time.\\
