\chapter{Introduction}
\section{Overview}
%takes p.78~ of ETD Style Manual
``The theory of hypergraphs is seen to be a very useful tool for the solution of integer optimization problems when the matrix has certain special properties'' - \cite{berge1984hypergraphs}

The enumeration of all minimal traversals of a hypergraph is  defined in \cite{berge1984hypergraphs}. The problem has already been shown to be computationally equivalent to the NP-Complete problem space. \cite{eiter1991transveral} This paper defines the minimal hypergraph transversals problem and demonstrates an iterative polynomial space solution. The previous recursive solution \cite{kavvadias2005efficient} was used as inspiration for the design of this solution.\\

The following definitions are used extensively throughout the paper: list, odometer, hyperedge, $unrestricted$ hypergraph, $normal$ hypergraph, $simple$ hypergraph, generalized variable, minimal hypergraph transversal, and partial hypergraph transversals.\\


Hypergraphs are a recent mathematical and computational discovery. The complexity of hypergraphs should not be underestimated. The number of edges in a $normal$ hypergraph is potentially $2^N$. Abstracting from a $normal$ to a $unrestricted$ hypergraph allows the edges to grow unbounded $N^\infty$. Even in this form an $unrestricted$ hypergraph can still be reasoned about. Odometers \cite{Odometer:Fuchs} are introduced as construct that can be useful for reasoning about hypergraphs. \\


The paper introduces equivalences between an odometer and hyperedge given a hypergraph. Additionally odometers are used as generalized variables, and lists of odometers as partial hypergraph transversals. Fundamental operations on odometers are included in the appendix for completeness. Additionally odometers are a structure that is independent of hypergraphs that can be reasoned about in their own context and worthy of their own research.\\

Using only these definitions, an algorithm that enumerates the minimal transversals is introduced. There are multiple corresponding NP-Complete problems that are shown to be equivalent \cite{eiter1995identifying}. Finding all minimal transversals is a significant and worthy problem in computation and especially in AI. \cite{reiter1987theory}, \cite{de1987diagnosing}

There are objectively better minimal transversals, a simple example is minimizing the total count of vertexes in the traversal \cite{bailey2003fast}. The first step to the enumeration of objectively better minimal transversals, is first traversing ONLY the minimal traversals in an efficient way, \cite{boros2003efficient}. This paper introduce a new iterative polynomial space algorithm to numerate all minimal hypergraph transversals efficiently.

\section{Lists, Queues, Stacks, N-Trees, etc...}
\begin{definition}
	Let a list $l$ be an ordered list of things $\{t,i\}$ where each thing $t$ is of the same type, and $t$ is only distinguishable by its index $i$.
\end{definition}
This paper assumes the reader is familiar with the basic data structures such as lists, arrays, stacks, queues, trees, graphs etc. 

For the purposes of this paper, $x.push(y)$ will insert $y$ at the last index of the list, $y \gets x.pop()$ will remove from the last index of the list and assign it to the variable $y$.  $x.enque(y)$ will insert $y$ at the last index of the list. $y \gets x.dequeue()$ will remove from the first index of the list and assign it to the variable $y$. In these example $x$ has an associated type, $y$ is a value of that same type in all cases. A list-of-lists structure is used in this paper to traverse an N-way branching tree. The internal C++ implementation is an array representation with $O(1)$ index lookup time, $O(N)$ seek, insert, and remove times.