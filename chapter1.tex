\chapter{Introduction}

%takes p.78~ of ETD Style Manual
``The theory of hypergraphs is seen to be a very useful tool for the solution of integer optimization problems when the matrix has certain special properties'' - \cite{berge1984hypergraphs}

Hypergraphs are a recent mathematical and computational discovery. The number of edges in a $normal$ hypergraph is potentially $2^N$. Abstracting from a $normal$ to a $unrestricted$ hypergraph allows the edges to unbounded $N^\infty$. An $unrestricted$ hypergraph can still be reasoned about. Odometers \cite{Odometer:Fuchs} are defined and the functions that operate on them. Then hyperedges are shown to be mutually interchangeable with odometers given a hypergraph and transformation functions. \\

The NP-Complete problem enumeration of all minimal traversals of a hypergraph is  defined \cite{eiter1991transveral}. There are multiple corresponding NP-Complete problems that are shown to be equivalent \cite{eiter1995identifying}. Finding all minimal transversals is a significant and worthy problem in computation and especially in AI \cite{eiter2002hypergraph}. 

There are objectively better minimal transversals such as the total count of vertexes in the traversal \cite{bailey2003fast}. Traversing a hypergraph is the first step to efficient traversal \cite{boros2003efficient}. Optimal traversals of a hypergraph are fundamentally similar to optimization of a property and NP-Hard. It is possible to at least generate efficient traversals that generate . \\ 