\chapter{Future Directions}

\section{Distributed Computation via Iterative Containers}
Fundamentally this paper re-introduces a polynomial space algorithm for computing hypergraph transversals as an iterative procedure. Each work item that is processed is a partial transversal with the included negative sets and a new hyperedge. The abstraction completely captures all the data needed to generate the next set of work items. The N-Way-Tree algorithm that iteratively constructs the next work items or processes completed transversals can be easily replaced.\\

The replacement for the N-Way-Tree would be a distributed processing dispatcher. Each work item would be put in a queue for execution on a compute node. All data to process the node is present in the work item. All the future work items can be generated on a compute node, then enqueued in the distributed dispatcher processor. \\

Hyperedges are always encountered in the same order for all work items. A pipeline of compute nodes with hyperedges can be built to distribute the work. A compute node is paired with a specific hyperedge and a work queue.\\

As each work item would be dequeued on the current compute node with the current edge. Work items would be generated and enqueued in the next compute node with the next edge. Controlling work item generation to keep the pipeline full becomes a new problem.\\


\section{Hyperedge Visitation Manipulation}
Conceptually a compute node contains a queue of work items and a hyperedge, an interesting and notable effect of encountering a new hyperedge when the partial transversal negates all new transversals, causing the complete removal of all future work items. Simply a new hyperedge is encountered that causes a minimal transversal to either become invalid or redundant. Eliminating future minimal transversals before they generate children that need to be eliminated is a possible optimization. \\

Reordering the visitation of hyperedges would need to be provably sound such that an dispatch algorithm could reorder the visitation of hyperedges to eliminate minimal transversals as quickly as possible during computation. As each work item has a list of negation odometers, it is possible to look for a hyperedge contained in the negation odometer union with the transversal odometer to generate either 1 new minimal transversal child or not being an appropriate minimal transversal and being eliminated from the work queue. \\




\section{Advanced Algorithmic Modifications}
The inter-section, outer-section, and cross product sections each generate a work item. Currently all work items for a given level are expanded to the next level (exponential in the worst case), then processing the first one in the same way expanding the tree in a depth first search. Control of the expansion can be done via additional odometer states in each work item. Replacement of the function $Gen2expNtruefalse$ with an iterative generator allows for a more controlled expansion. \\

The generation of the next list of work items is currently a fixed ordering that generates the same traversals in the same order every time. Obviously this is desirable from a completeness aspect, but from a practical perspective iterating the traversals which satisfy optimality constraints `better' is of particular interest. `Better' is a term that needs to be defined in terms of the particular problem that is being solved. \cite{khachiyan2006efficient}\\




\section{Experimental Research}
A possible direction of research is to modify the core algorithms to look for ways to collapse the work items being generated at each depth. The compact transversal representations generated by this algorithm store exponential traversals in polynomial space. Removal of the negation sets in conjunction with collapsing work items at each level would result in a polynomial reduction by storing traversals in an exponential encoding. This is the partial transversal frame that is enumerated to generate its transversals \\

It is not possible to enumerate all transversals of a hypergraph in polynomial space and polynomial time. It may be possible to enumerate all exponential encodings of transversals of a hypergraph in polynomial space and polynomial time. The current algorithm already generates a polynomial space encoding of an exponential transversal. it breaks the encoding down 

a representation of transversals as a polynomial space encoding that expands into an exponential number of transversals.


